\section{Writing Hexcode in Disassembler}
\begin{itemize}


\item \textbf{STEP 1:} To Enter the hexcode\\

\begin{tabular}{cccccc}
\textbf{<Start Code>} & \textbf{<Byte Count>} & \textbf{<Address>} & <\textbf{Record Type>} & \textbf{<Data>} & \textbf{<Checksum>}\\
: & 10 & 0000 & 00 & Enter 10 bytes  & <ctrl+space>\\
&&&& of data in Hexadecimal&\\
&&&& format &
\end{tabular}

\item \textbf{STEP 2:} To mark end of file\\

\begin{tabular}{cccccc}
\textbf{<Start Code>} & \textbf{<Byte Count>} & \textbf{<Address>} & <\textbf{Record Type>} & \textbf{<Data>} & \textbf{<Checksum>}\\
: & 00 & 0000 & 01 &  & FF
\end{tabular}

\end{itemize}

\paragraph{TOOLS EMBEDDED IN DISASSEMBLER EDITOR}
\begin{itemize}
\item \textbf{\textit{AUTO CHECKSUM GENERATION}} \\ Just press \textbf{CTRL+SPACE} at the end of each line it is auto calculated and appended to that line
\item \textbf{\textit{AUTO SYNTAX HIGHLIGHTING and FORMATING}} \\It is activated on pressing of \textbf{ENTER} key.  
\end{itemize}

\subsection{Limitation of disassembler}
\begin{itemize}
\item Cannot determine the begin address of execution
\item Fails to distinguish between user defined data code and opcode, so it by default decode all as opcode.
\end{itemize} 